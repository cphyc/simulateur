\documentclass[a4paper]{article}

\usepackage[french]{babel}
\usepackage[T1]{fontenc}
\usepackage[utf8]{inputenc}
\usepackage{fullpage}
\usepackage{pgf}
\usepackage{tikz}
\usetikzlibrary{arrows,automata}
\usetikzlibrary{positioning,calc}

\tikzstyle{operateur} =
[style={circle,fill=gray!20,draw,font=\sffamily\large\bfseries}]
\tikzstyle{not} = [draw,style={circle,fill=gray!20},minimum size=0.0cm]
\tikzstyle{reg} = [style={fill=gray!20,draw,thick,minimum size=0.8cm}]
\tikzstyle{boite} = [minimum height=2cm, minimum width=2cm,rectangle,
					style={fill=gray!20,draw,thick}]
\tikzstyle{rambox} = [minimum height=7cm, minimum width=2cm,rectangle,
					style={fill=gray!20,draw,thick}]
\tikzstyle{muxv} = [minimum height=1.6cm, minimum width=0.5cm,rectangle,
	               style={fill=gray!20,draw,thick}]
\tikzstyle{mux} = [minimum height=0.5cm, minimum width=1.6cm,rectangle,
	               style={fill=gray!20,draw,thick}]
\tikzstyle{muxb} = [node distance=0.4cm,style={minimum size=0.5cm}]
\tikzstyle{vide} = []
\tikzstyle{noeud} = [style={circle,fill=black,minimum size=0cm}]
\tikzstyle{fil} = [->,style={thick}]
\tikzstyle{filb} = [style={thick}]
\tikzstyle{ecrit} = [font=\sffamily\small\bfseries]
\tikzstyle{exterieur} = [dashed, draw]
\tikzstyle{hypot} = [dashed]

\begin{document}

\title{Microprocessor design}
\author{Baptiste Lefebvre, Li-yao Xia, Antonin Delpeuch}
\date{\today}

\maketitle

\section{Motivation}

We wanted to design an original microprocessor. The requirements were :
\begin{itemize}
\item Be able to run a digital watch ;
\item Use an uncommon architecture, contrasting with the MIPS standard as we
already discovered it in the compilation course ;
\item Be able to run arbitrary programs.
\end{itemize}

Reduced Instruction Set Chips (such as ARM) are widely used, but their instruction set
is not really small. A natural question arises : to what extent can we
reduce this set ? 

\section{Instruction set}

We discovered the Subleq instruction set, which is made of an only
instruction, called \texttt{subleq}. Here is the behaviour of
\texttt{subleq a b c} :

\begin{verbatim}
b := b - a;
if b > 0 then
    go to the next instruction
else go to c
\end{verbatim}




\section{Chip}

\begin{figure}
   \centering
    \begin{tikzpicture}[auto, node distance=1cm,>=latex']
    
    \node[not] at (-3cm,-1cm) (n1) {};
    \node[reg, node distance=0.2cm, right=of n1] (r1) {st1};
    \node[not, node distance=0.2cm, right=of r1] (n2) {};
    \node[reg, right=of r1] (r2) {st2};
    \node[reg, right=of r2] (r3) {st3};
    \node[reg, right=of r3] (r4) {st4};
    \node[reg, right=of r4] (r5) {st5};

    \node[above right of= r5] (ctrl1) {};
    \node[above left of= n1] (ctrl2) {};
    \draw[fil] (n1) -- (r1);
    \draw[fil] (r1) -- (n2);
    \draw[fil] (n2) -- (r2);
    \draw[fil] (r2) -- (r3);
    \draw[fil] (r3) -- (r4);
    \draw[fil] (r4) -- (r5);
    \draw[filb] (r5) -| (ctrl1.mid);
    \draw[filb] (ctrl1.mid) -- (ctrl2.mid);
    \draw[fil] (ctrl2.mid) |- (n1);

    \foreach \i in {0,1,2} {
	\node[reg] at ($(0,-5)+3*(0.5+\i,0)$) (rn\i)  {};
	\node[mux, below of=rn\i, node distance=1.5cm] (mn\i)
         {0\hspace{0.6cm}1};
	\node[below left of=mn\i, node distance=1cm] (ctrlrop\i) {};
	\node[below right of=mn\i, node distance=1.5cm] (ctrlroq\i) {};
	\node[above left of=rn\i, node distance=1.5cm] (ctrlal\i) {};
	\draw[fil] (mn\i) -- (rn\i);
	\draw[fil] (ctrlrop\i.mid) -|  ($(mn\i)+(-0.4cm,-0.25cm)$);
	\draw[filb] (ctrlrop\i.mid) -| (ctrlal\i.mid);
	\draw[filb] (ctrlal\i.mid) -| (rn\i);
	\draw[fil] (ctrlroq\i.mid) -| ($(mn\i)+(0.4cm,-0.25cm)$); 
	}
   
    \node at ($(ctrlroq2)+(4cm,0)$) (ctrlrr) {};
    \draw[filb] (ctrlroq2.mid) -- (ctrlrr.mid);
    \draw[filb] (ctrlroq0.mid) -- (ctrlroq1.mid) -- (ctrlroq2.mid);

    \node[rambox] at ($(ctrlrr)+(-2cm,7cm)$) (rammod) {Memory}; 
    \draw[filb] (ctrlrr.mid) |- (rammod);

    \node[reg] at (0,4) (pcreg) {};
    \node[muxv] at
($(pcreg)-(1.5cm,0.4cm)$) (pcmuxi) {};
    \node[muxv, node distance=3cm, left of=pcreg] (pcmux) {};
    \node at ($(pcmux)+(0,0.4cm)$) (pcmux0) {0};
    \node at ($(pcmux)+(0,-0.4cm)$) (pcmux1) {1};
    \node at ($(pcmuxi)+(0,0.4cm)$) (pcmuxi0) {0};
    \node at ($(pcmuxi)+(0,-0.4cm)$) (pcmuxi1) {1};
    \node[node distance=1.5cm, above right of=pcreg] (pcrct) {};
    \node[node distance=1.3cm, above left of=pcmux] (pcl1ct) {};
    \node[node distance=1.7cm, above left of=pcmux] (pcl2ct) {};
    \node[operateur, node distance=2cm, above of=pcreg] (incr1) {+1};
    \draw[fil] (pcmux) -- ($(pcmuxi)+(-0.25,0.4cm)$);
    \draw[filb] (pcreg) -| (pcrct.mid) -| (pcl1ct.mid);
    \draw[fil] (pcl1ct.mid) |- ($(pcmux)+(-0.25cm,0.4cm)$);
    \draw[fil] (pcrct.mid) |- (incr1);
    \draw[fil] (incr1) -| (pcl2ct.mid) |- ($(pcmux)+(-0.25cm,-0.4cm)$);
    
    \node[operateur, node distance=2cm, below of=pcmux] (pcunion) {U};
    \draw[filb] ($(pcunion)+(-0.5cm,-0.5cm)$) -- (pcunion) ++(-0.6cm,-0.7cm)
node {st1};
    \draw[filb] ($(pcunion)+(0cm,-0.5cm)$) -- (pcunion) ++(0,-0.7cm)
node {st2};
    \draw[filb] ($(pcunion)+(0.5cm,-0.5cm)$) -- (pcunion) ++(0.6cm,-0.7cm)
node {st5};
    \draw[fil] ($(pcunion)+(0.5cm,-0.5cm)$) -- (pcunion);
    \draw[fil] ($(pcunion)+(0cm,-0.5cm)$) -- (pcunion);
    \draw[fil] ($(pcunion)+(-0.5cm,-0.5cm)$) -- (pcunion);
    \draw[fil] (pcunion) -- (pcmux);
    \draw[fil] ($(pcmuxi)+(0.25cm,0.4cm)$) -- (pcreg);

    \node[boite] at ($(rammod)+(-3.5cm,2cm)$) (muxers) {Muxers};
    \draw[fil] ($(muxers)+(1cm,0.6cm)$) -- ($(rammod)+(-1cm,2.6cm)$);
    \draw[filb] ($(muxers)+(1cm,0.6cm)$) -- ($(rammod)+(-1cm,2.6cm)$)
         ++(-0.9cm,0.15cm) node {wa};
    \draw[fil] ($(muxers)+(1cm,0.2cm)$) -- ($(rammod)+(-1cm,2.2cm)$);
    \draw[filb] ($(muxers)+(1cm,0.2cm)$) -- ($(rammod)+(-1cm,2.2cm)$)
         ++(-0.9cm,0.15cm) node {ra};
    \draw[fil] ($(muxers)+(1cm,-0.2cm)$) -- ($(rammod)+(-1cm,1.8cm)$);
    \draw[filb] ($(muxers)+(1cm,-0.2cm)$) -- ($(rammod)+(-1cm,1.8cm)$)
         ++(-0.9cm,0.15cm) node {wen};
    \draw[fil] ($(muxers)+(1cm,-0.6cm)$) -- ($(rammod)+(-1cm,1.4cm)$);
    \draw[filb] ($(muxers)+(1cm,-0.6cm)$) -- ($(rammod)+(-1cm,1.4cm)$)
         ++(-0.9cm,0.15cm) node {data};

    \end{tikzpicture}
    \caption{Chip's architecture}
\end{figure}



\end{document}



\documentclass{beamer}
\usepackage[utf8]{inputenc}
\usepackage[francais]{babel}
\usepackage{bbm}
\usepackage{epstopdf}
\usepackage{pdfpages}
\usepackage{graphicx}
\usepackage{amsthm}
\usepackage{amsmath}
\usetheme{default}
\usecolortheme{seahorse}
\usepackage{mathtools}
\usepackage{listings}
\usepackage{tikz}
\usetikzlibrary{arrows,automata,calc,decorations.pathmorphing,backgrounds,positioning}

\beamertemplatenavigationsymbolsempty
\setbeamertemplate{navigation symbols}{} 

\usepackage[T1]{fontenc}


\hypersetup{pdfstartview={FitH}}


\begin{document}

\makeatletter

\title{A Watch Using Subleq}
\author{Baptiste Lefebvre, Li-yao Xia, Antonin Delpeuch}
\date{2013/01/22 15 : 55}

\begin{frame}
    \maketitle
\end{frame}

\begin{frame}
    \frametitle{Outline}
    \tableofcontents
\end{frame}

\section{The \texttt{Subleq} language}
\subsection{Motivation}

\begin{frame}
    \frametitle{Motivation}

    \begin{itemize}
        \item We discovered the MIPS architecture in compilation course.
        \item We wanted to design an original processor.
        \item We focused on environments in which the number of gates must be kept small.
    \end{itemize}

\end{frame}

\subsection{Definition}
\begin{frame}[fragile]
    \frametitle{\texttt{Subleq}}

    The instruction set is made of only one instruction written as 3
    integers :
    \\[11pt]
    \begin{verbatim}
      a b c
    \end{verbatim}
    which means :

    \begin{verbatim}
      b <- *b - *a
      \textbf{if} b <= 0 then go to c
      else go to next instruction
    \end{verbatim}

\end{frame}

\subsection{Examples}
\begin{frame}[fragile]{Examples}
  \begin{verbatim}
    3  4  6   8  8  -1   5  3  0
    3  4  6   8  0  -1   5  3  0
    3  4  6   9  0  -1   5  3  0
    3  4  6   9 -9  -1   5  3  0
    3  4  6  10 -9  -1   5  3  0
  \end{verbatim}
\end{frame}

\begin{frame}[fragile]
    \frametitle{Examples}
Common instructions can be written with \texttt{Subleq}, for instance :

\begin{columns}[t]
    \begin{column}[T]{3cm}
\begin{verbatim}
move a b :
  a Z ;
  b b ;
  Z b ;
  Z Z ;

add a b c :
  c Z ;
  b Z ;
  a a ;
  Z a ;
  Z Z ;
\end{verbatim}
    \end{column}
\begin{column}[T]{3cm}
\begin{verbatim}
beqz a lbl :
   Z a L1
   Z Z L3
L1:
   a Z L2
   Z Z L3
L2:
   Z Z lbl
L3:

j lbl :
   Z Z lbl
\end{verbatim}
\end{column}
\end{columns}
\end{frame}

\section{Chip architecture}
\subsection{The cycle}

\begin{frame}[fragile]
    \frametitle{The cycle}
\begin{columns}[t]
	\begin{column}[T]{4cm}
    \begin{enumerate}
      \setcounter{enumi}{0}
        \item \textbf{Step 1}
          \begin{verbatim}
Store A
Read address : B
Increment PC
          \end{verbatim}
        \item \textbf{Step 2}
          \begin{verbatim}
Store B
Read address : VA
          \end{verbatim}
        \item \textbf{Step 3}
	        \begin{verbatim}
Store VA in A
Read address : VB
          \end{verbatim}
  	\end{enumerate}
	\end{column}
	\begin{column}[T]{7cm}
  	\begin{enumerate}
  	  \setcounter{enumi}{3}
  	    \item \textbf{Step 4}
  	      \begin{verbatim}
Store VB
Read address : C
Increment PC
  	      \end{verbatim}
  	    \item \textbf{Step 5}
  	      \begin{verbatim}
If VB - VA <= 0 then set PC to C
Read address : A at PC
Increment PC
  	      \end{verbatim}
  	\end{enumerate}
	\end{column}
\end{columns}

\end{frame}

\subsection{Outline of the chip}
\begin{frame}
    \frametitle{Outline}
    \centering
    
\tikzstyle{operateur} =
[style={circle,fill=gray!20,draw,font=\sffamily\large\bfseries}]
\tikzstyle{not} = [draw,style={circle,fill=gray!20},minimum size=0.0cm]
\tikzstyle{reg} = [style={fill=gray!20,draw,thick,minimum size=0.8cm}]
\tikzstyle{boite} = [minimum height=2cm, minimum width=2cm,rectangle,
					style={fill=gray!20,draw,thick}]
\tikzstyle{ptiteboite} = [minimum height=1.5cm, minimum width=1.5cm,rectangle,
					style={fill=gray!20,draw,thick}]
\tikzstyle{rambox} = [minimum height=7cm, minimum width=2cm,rectangle,
					style={fill=gray!20,draw,thick}]
\tikzstyle{muxv} = [minimum height=1.6cm, minimum width=0.5cm,rectangle,
	               style={fill=gray!20,draw,thick}]
\tikzstyle{mux} = [minimum height=0.5cm, minimum width=1.6cm,rectangle,
	               style={fill=gray!20,draw,thick}]
\tikzstyle{muxb} = [node distance=0.4cm,style={minimum size=0.5cm}]
\tikzstyle{vide} = []
\tikzstyle{noeud} = [style={circle,fill=black,minimum size=0cm}]
\tikzstyle{fil} = [->,style={thick}]
\tikzstyle{filb} = [style={thick}]
\tikzstyle{ecrit} = [font=\sffamily\small\bfseries]
\tikzstyle{exterieur} = [dashed, draw]
\tikzstyle{hypot} = [dashed]

\begin{tikzpicture}[auto, node distance=3.5cm,>=latex', scale = 0.75, every node/.style={transform shape}]

\node[boite] (muxers) {Muxers};

\node[boite, right of=muxers] (memctrl) {Memory controller};
\node[boite, above right of=memctrl] (ram) {RAM};
\node[ptiteboite, below right of=memctrl] (dd) {DD};
\node[ptiteboite, right of=dd, node distance=2.5cm] (bcd) {BCD};

\node[reg, above of=muxers] (pc) {PC};
\node[reg, below of=muxers] (b) {B};
\node[reg, left of=b,node distance=1cm] (a) {A};
\node[reg, right of=b,node distance=1cm] (vb) {VB};
\node[boite, left of=muxers] (sched) {Scheduler};
\node[operateur, below of=b, node distance=2cm] (sub) {sub};
\node[operateur, left of=sub] (leqz) {leqz};

\draw[fil] (sched) -- (muxers);
\draw[filb] (muxers) -- (memctrl);
\draw[filb] (pc) -- (muxers);
\draw[filb] (a) |- ($(a) + (0.8cm,0.6cm)$) -| ($(muxers)+(-0.2cm,-1cm)$);
\draw[filb] (b) -- (muxers);
\draw[filb] (vb) |- ($(vb) + (-0.8cm,0.6cm)$) -| ($(muxers)+(0.2cm,-1cm)$);

\draw[fil] (memctrl) |- (dd);
\draw[fil] (dd) -- (bcd);
\draw[filb] (memctrl) |- (ram);

\draw[fil] (a) |- ($(a)+(0.4cm,-0.6cm)$) -| ($(sub)+(-0.4cm,0.4cm)$);
\draw[fil] (vb) |- ($(vb)+(-0.4cm,-0.6cm)$) -| ($(sub)+(0.4cm,0.4cm)$);
\draw[fil] (sub) -- (leqz);
\draw[fil] (leqz) -| ($(leqz)+(1.8cm,3cm)$) -| ($(muxers)+(-0.4cm,-1cm)$);
\draw[fil] (leqz) |- ($(leqz)+(1.8cm,3.2cm)$) -| ($(muxers)+(-0.6cm,-1cm)$);


\end{tikzpicture}


\end{frame}

\section{Implementation details}

\subsection{Memory management}

\begin{frame}
e
\end{frame}

\subsection{7-segments output}

\begin{frame}
\frametitle{7-segments output}
The conversion from binary values to 7-segments output is built in the chip, in two steps :
\begin{enumerate}
    \item Binary $\rightarrow$ Binary Coded Decimal (BCD)
        
          \texttt{101010} $\rightarrow$ \texttt{0100}.\texttt{0010}
    \item BCD $\rightarrow$ 7-segments

        \texttt{0100}.\texttt{0010} $\rightarrow$ \texttt{1100110}.\texttt{0111011}
\end{enumerate}

The first step is done with the \emph{double dabble} algorithm, the second step using a circuit derived
from the desired truth table.

\end{frame}

\begin{frame}
    \frametitle{Double dabble}

    How to multiply by 2 a number written in BCD ?

    If the digits are all less than 5, one shift is enough.
    Adding a carry if a digit is $\geq$ 5 is necessary
\end{frame}

\end{document}


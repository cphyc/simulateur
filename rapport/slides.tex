\documentclass[slidestop]{beamer}
\usepackage[utf8]{inputenc}
\usepackage[francais]{babel}
\usepackage{bbm}
\usepackage{epstopdf}
\usepackage{pdfpages}
\usepackage{graphicx}
\usepackage{amsthm}
\usepackage{amsmath}
\usetheme{Boadilla}
\usecolortheme{seahorse}
\usepackage{mathtools}
\usepackage{listings}
\usepackage{tikz}
\usetikzlibrary{arrows,automata,calc,decorations.pathmorphing,backgrounds,positioning}

\beamertemplatenavigationsymbolsempty
\setbeamertemplate{navigation symbols}{} 

\usepackage[T1]{fontenc}


\hypersetup{pdfstartview={FitH}}
\begin{document}

\makeatletter

\title{A Watch Using Subleq}
\author{Baptiste Lefebvre, Li-yao Xia, Antonin Delpeuch}
\date{2013/01/22 15 : 55}

\begin{frame}
    \maketitle
\end{frame}

\begin{frame}
    \frametitle{Outline}
    \tableofcontents
\end{frame}
\section{The \texttt{subleq} language}
\subsection{Motivation}

\begin{frame}
    \frametitle{Motivation}

    \begin{itemize}
        \item We discovered the MIPS architecture in compilation course.
        \item We wanted to design an original processor.
        \item We focused on environments in wich the number of gates must be kept small.
    \end{itemize}

\end{frame}

\subsection{Definition}
\begin{frame}[fragile]
    \frametitle{\texttt{subleq}}

    The instruction set is made of only one instruction :
    \\[11pt]
    \begin{lstlisting}
      a b c
    \end{lstlisting}
    which means :

    \begin{lstlisting}
      b := b - a;
      if b > 0 then
        go to the next instruction
      else go to c
    \end{lstlisting}

\end{frame}

\subsection{Examples}
\begin{frame}[fragile]
    \frametitle{Examples}
Common instructions can be written with \texttt{subleq}, for instance :

\begin{columns}[t]
    \begin{column}[T]{5cm}
\begin{lstlisting}
move a b :
  a Z ;
  b b ;
  Z b ;
  Z Z ;

add a b c :
  c Z ;
  b Z ;
  a a ;
  Z a ;
  Z Z ;
\end{lstlisting}
    \end{column}
\begin{column}[T]{5cm}
\begin{lstlisting}
beqz a lbl :
   Z a L1
   Z Z L3
L1:
   a Z L2
   Z Z L3
L2:
   Z Z lbl
L3:

j lbl :
   Z Z lbl
\end{lstlisting}
\end{column}
\end{columns}
\end{frame}

\section{Chip architecture}
\subsection{The cycle}

\begin{frame}[fragile]
    \frametitle{The cycle}

    \begin{columns}[t]
        \begin{column}[T]{5cm}
    \begin{enumerate}
            \setcounter{enumi}{-1}

        \item \textbf{Step 1}
            \begin{lstlisting}
 Store A,
 Ask for B,
 Increment PC
            \end{lstlisting}
        \item \textbf{Step 2}
            \begin{lstlisting}
 Store B,
 Ask for VA
            \end{lstlisting}

        \item \textbf{Step 3}
            \begin{lstlisting}
 Store VA in A,
 Ask for VB
            \end{lstlisting}

    \end{enumerate}
\end{column}
\begin{column}[T]{5cm}
    \begin{enumerate}
        \setcounter{enumi}{1}
        \item \textbf{Step 4}
            \begin{lstlisting}
 Store VB,
 Ask for C,
 Increment PC
            \end{lstlisting}

        \item \textbf{Step 5}
            \begin{lstlisting}
 If VB - VA <= 0
   then set PC to C,
 Ask for A at PC,
 Increment PC
             \end{lstlisting}
    \end{enumerate}
\end{column}
\end{columns}

\end{frame}

\subsection{Outline of the chip}
\begin{frame}
    \frametitle{Outline}
    
\tikzstyle{operateur} =
[style={circle,fill=gray!20,draw,font=\sffamily\large\bfseries}]
\tikzstyle{not} = [draw,style={circle,fill=gray!20},minimum size=0.0cm]
\tikzstyle{reg} = [style={fill=gray!20,draw,thick,minimum size=0.8cm}]
\tikzstyle{boite} = [minimum height=2cm, minimum width=2cm,rectangle,
					style={fill=gray!20,draw,thick}]
\tikzstyle{rambox} = [minimum height=7cm, minimum width=2cm,rectangle,
					style={fill=gray!20,draw,thick}]
\tikzstyle{muxv} = [minimum height=1.6cm, minimum width=0.5cm,rectangle,
	               style={fill=gray!20,draw,thick}]
\tikzstyle{mux} = [minimum height=0.5cm, minimum width=1.6cm,rectangle,
	               style={fill=gray!20,draw,thick}]
\tikzstyle{muxb} = [node distance=0.4cm,style={minimum size=0.5cm}]
\tikzstyle{vide} = []
\tikzstyle{noeud} = [style={circle,fill=black,minimum size=0cm}]
\tikzstyle{fil} = [->,style={thick}]
\tikzstyle{filb} = [style={thick}]
\tikzstyle{ecrit} = [font=\sffamily\small\bfseries]
\tikzstyle{exterieur} = [dashed, draw]
\tikzstyle{hypot} = [dashed]

\begin{tikzpicture}[auto, node distance=1cm,>=latex', scale = 0.6, every node/.style={transform shape}]
 
    \node[not] at (4cm,4cm) (n1) {};
    \node[reg, node distance=0.2cm, right=of n1] (r1) {st1};
    \node[not, node distance=0.2cm, right=of r1] (n2) {};
    \node[reg, right=of r1] (r2) {st2};
    \node[reg, right=of r2] (r3) {st3};
    \node[reg, right=of r3] (r4) {st4};
    \node[reg, right=of r4] (r5) {st5};

    \node[above right of= r5] (ctrl1) {};
    \node[above left of= n1] (ctrl2) {};
    \draw[fil] (n1) -- (r1);
    \draw[fil] (r1) -- (n2);
    \draw[fil] (n2) -- (r2);
    \draw[fil] (r2) -- (r3);
    \draw[fil] (r3) -- (r4);
    \draw[fil] (r4) -- (r5);
    \draw[filb] (r5) -| (ctrl1.mid);
    \draw[filb] (ctrl1.mid) -- (ctrl2.mid);
    \draw[fil] (ctrl2.mid) |- (n1);

	\node[reg] at ($(0,-5)+3*(0.5+0,0)$) (rn0)  {A};
	\node[mux, below of=rn0, node distance=1.5cm] (mn0)
         {0\hspace{0.6cm}1};
	\node[below left of=mn0, node distance=1cm] (ctrlrop0) {};
	\node[below right of=mn0, node distance=1.5cm] (ctrlroq0) {};
    \node at ($(rn0)+(-1cm,0.6cm)$) (ctrlal0) {};
	\draw[fil] (mn0) -- (rn0);
	\draw[fil] (ctrlrop0.mid) -|  ($(mn0)+(-0.4cm,-0.25cm)$);
	\draw[filb] (ctrlrop0.mid) -| (ctrlal0.mid);
	\draw[filb] (ctrlal0.mid) -| (rn0);
	\draw[fil] (ctrlroq0.mid) -| ($(mn0)+(0.4cm,-0.25cm)$); 


   	\node[reg] at ($(0,-5)+3*(0.5+1,0)$) (rn1)  {B};
	\node[mux, below of=rn1, node distance=1.5cm] (mn1)
         {0\hspace{0.6cm}1};
	\node[below left of=mn1, node distance=1cm] (ctrlrop1) {};
	\node[below right of=mn1, node distance=1.5cm] (ctrlroq1) {};
	\node at ($(rn1)+(-1cm,0.6cm)$) (ctrlal1) {};
	\draw[fil] (mn1) -- (rn1);
	\draw[fil] (ctrlrop1.mid) -|  ($(mn1)+(-0.4cm,-0.25cm)$);
	\draw[filb] (ctrlrop1.mid) -| (ctrlal1.mid);
	\draw[filb] (ctrlal1.mid) -| (rn1);
	\draw[fil] (ctrlroq1.mid) -| ($(mn1)+(0.4cm,-0.25cm)$); 


	\node[reg] at ($(0,-5)+3*(0.5+2,0)$) (rn2)  {VB};
	\node[mux, below of=rn2, node distance=1.5cm] (mn2)
         {0\hspace{0.6cm}1};
	\node[below left of=mn2, node distance=1cm] (ctrlrop2) {};
	\node[below right of=mn2, node distance=1.5cm] (ctrlroq2) {};
	\node at ($(rn2)+(-1cm,0.6cm)$)  (ctrlal2) {};
	\draw[fil] (mn2) -- (rn2);
	\draw[fil] (ctrlrop2.mid) -|  ($(mn2)+(-0.4cm,-0.25cm)$);
	\draw[filb] (ctrlrop2.mid) -| (ctrlal2.mid);
	\draw[filb] (ctrlal2.mid) -| (rn2);
	\draw[fil] (ctrlroq2.mid) -| ($(mn2)+(0.4cm,-0.25cm)$); 

    \node at ($(ctrlroq2)+(4cm,0)$) (ctrlrr) {};
    \draw[filb] (ctrlroq2.mid) -- (ctrlrr.mid);
    \draw[filb] (ctrlroq0.mid) -- (ctrlroq1.mid) -- (ctrlroq2.mid);

    \node[rambox] at ($(ctrlrr)+(-2cm,7cm)$) (rammod) {Memory}; 
    \draw[filb] (ctrlrr.mid) |- (rammod);

    \node[reg] at (2,0) (pcreg) {pc};
    \node[muxv] at
($(pcreg)-(1.5cm,0.4cm)$) (pcmuxi) {};
    \node[muxv, node distance=3cm, left of=pcreg] (pcmux) {};
    \node at ($(pcmux)+(0,0.4cm)$) (pcmux0) {0};
    \node at ($(pcmux)+(0,-0.4cm)$) (pcmux1) {1};
    \node at ($(pcmuxi)+(0,0.4cm)$) (pcmuxi0) {0};
    \node at ($(pcmuxi)+(0,-0.4cm)$) (pcmuxi1) {1};
    \node at ($(pcmuxi)+(-0.5cm,-0.4cm)$) {c};
    \node[node distance=1.5cm, above right of=pcreg] (pcrct) {};
    \node[node distance=1.3cm, above left of=pcmux] (pcl1ct) {};
    \node[node distance=1.7cm, above left of=pcmux] (pcl2ct) {};
    \node[operateur, node distance=2cm, above of=pcreg] (incr1) {+1};
    \draw[fil] (pcmux) -- ($(pcmuxi)+(-0.25,0.4cm)$);
    \draw[filb] (pcreg) -| (pcrct.mid) -| (pcl1ct.mid);
    \draw[fil] (pcl1ct.mid) |- ($(pcmux)+(-0.25cm,0.4cm)$);
    \draw[fil] (pcrct.mid) |- (incr1);
    \draw[fil] (incr1) -| (pcl2ct.mid) |- ($(pcmux)+(-0.25cm,-0.4cm)$);
    
    \node[operateur, node distance=1.5cm, below of=pcmux] (pcunion) {U};
    \draw[filb] ($(pcunion)+(-0.5cm,-0.5cm)$) -- (pcunion) ++(-0.6cm,-0.7cm)
node {st1};
    \draw[filb] ($(pcunion)+(0cm,-0.5cm)$) -- (pcunion) ++(0,-0.7cm)
node {st2};
    \draw[filb] ($(pcunion)+(0.5cm,-0.5cm)$) -- (pcunion) ++(0.6cm,-0.7cm)
node {st5};
    \draw[fil] ($(pcunion)+(0.5cm,-0.5cm)$) -- (pcunion);
    \draw[fil] ($(pcunion)+(0cm,-0.5cm)$) -- (pcunion);
    \draw[fil] ($(pcunion)+(-0.5cm,-0.5cm)$) -- (pcunion);
    \draw[fil] (pcunion) -- (pcmux);
    \draw[fil] ($(pcmuxi)+(0.25cm,0.4cm)$) -- (pcreg);

    \node[boite] at ($(rammod)+(-3.5cm,2cm)$) (muxers) {Muxers};
    \draw[fil] ($(muxers)+(1cm,0.6cm)$) -- ($(rammod)+(-1cm,2.6cm)$);
    \draw[filb] ($(muxers)+(1cm,0.6cm)$) -- ($(rammod)+(-1cm,2.6cm)$)
         ++(-0.9cm,0.15cm) node {wa};
    \draw[fil] ($(muxers)+(1cm,0.2cm)$) -- ($(rammod)+(-1cm,2.2cm)$);
    \draw[filb] ($(muxers)+(1cm,0.2cm)$) -- ($(rammod)+(-1cm,2.2cm)$)
         ++(-0.9cm,0.15cm) node {ra};
    \draw[fil] ($(muxers)+(1cm,-0.2cm)$) -- ($(rammod)+(-1cm,1.8cm)$);
    \draw[filb] ($(muxers)+(1cm,-0.2cm)$) -- ($(rammod)+(-1cm,1.8cm)$)
         ++(-0.9cm,0.15cm) node {wen};
    \draw[fil] ($(muxers)+(1cm,-0.6cm)$) -- ($(rammod)+(-1cm,1.4cm)$);
    \draw[filb] ($(muxers)+(1cm,-0.6cm)$) -- ($(rammod)+(-1cm,1.4cm)$)
         ++(-0.9cm,0.15cm) node {data};

    \draw[fil] (rn0) |- ++(4cm,1.5cm) |- ($(muxers)+(-1cm,0.5cm)$);
    \draw[fil] (rn1) |- ++(1.2cm,1.3cm) |- ($(muxers)+(-1cm,0.3cm)$);
    \draw[fil] (rn2) |- ++(-1.6cm,1.1cm) |- ($(muxers)+(-1cm,0.1cm)$);
    \draw[fil] (pcreg) -| ++(2cm,2cm) |- ($(muxers)+(-1cm,0.9cm)$);

    \node[operateur] at (-1cm,-5.2cm) (sub) {sub};
    \node[operateur] at (-1cm,-3.2cm) (neg) {neg};
    \draw[fil] (ctrlrr.mid) -| (sub);
    \draw[fil] (sub) -- (neg);
    \draw[filb] (sub) -| ++(1cm,1.5cm) -| ++(1.3cm,0.4cm) -- ++(4cm,0);
    \draw[fil] ($(sub)+(6.3cm,1.9cm)$) |- ($(muxers)+(-1cm,0.7cm)$);
    \draw[fil] (ctrlrop0.mid) -| ($(sub)+(0.35,-0.35)$);
    \draw[fil] (neg) -| (pcmuxi);

    \end{tikzpicture}


\end{frame}

\section{Implementation details}

\subsection{Memory management}

\begin{frame}
e
\end{frame}

\subsection{7-segments output}

\begin{frame}
\frametitle{7-segments output}
The conversion from binary values to 7-segments output is built in the chip, in two steps :
\begin{enumerate}
    \item Binary $\rightarrow$ Binary Coded Decimal (BCD)
        
          \texttt{101010} $\rightarrow$ \texttt{0100}.\texttt{0010}
    \item BCD $\rightarrow$ 7-segments

        \texttt{0100}.\texttt{0010} $\rightarrow$ \texttt{1100110}.\texttt{0111011}
\end{enumerate}
\end{frame}


\end{document}


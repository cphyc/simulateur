\documentclass[slidestop]{beamer}
\usepackage[utf8]{inputenc}
\usepackage[francais]{babel}
\usepackage{bbm}
\usepackage{epstopdf}
\usepackage{pdfpages}
\usepackage{graphicx}
\usepackage{amsthm}
\usepackage{amsmath}
\usetheme{Pittsburgh}
\usecolortheme{seagull}
\usepackage{mathtools}
\usepackage{stmaryrd}
\usepackage{minted}
\usepackage{tikz}
\usepackage{subfigure}
\usetikzlibrary{arrows,decorations.pathmorphing,backgrounds,positioning}

\beamertemplatenavigationsymbolsempty
\setbeamertemplate{navigation symbols}{} 

\usepackage[T1]{fontenc}


\hypersetup{pdfstartview={FitH}}
\begin{document}

\makeatletter

\title{A Watch Using Subleq}
\author{Baptiste Lefebvre, Li-yao Xia, Antonin Delpeuch}
\date{2013/01/22 15:55}

\section{The \texttt{subleq} language}

\begin{frame}
    \frametitle{Motivation}

    \begin{itemize}
        \item We discovered the MIPS architecture in compilation course.
        \item We wanted to design an original processor.
        \item We focused on environments in wich the number of gates must be reduced as much as possible.
    \end{itemize}

\end{frame}


\section{Chip architecture}

\section{Implementation details}

\subsection{Memory management}

\subsection{7-segments output}


\end{document}


\documentclass{beamer}
\usepackage[utf8]{inputenc}
\usepackage[francais]{babel}
\usepackage{bbm}
\usepackage{SevenSeg}
\usepackage{epstopdf}
\usepackage{pdfpages}
\usepackage{graphicx}
\usepackage{amsthm}
\usepackage{amsmath}
\usetheme{default}
\usecolortheme{seahorse}
\usepackage{mathtools}
\usepackage{tikz}
\usepackage{alltt}
\usetikzlibrary{arrows,automata,calc,decorations.pathmorphing,backgrounds,positioning}

%\beamertemplatenavigationsymbolsempty
%\setbeamertemplate{navigation symbols}{} 

\usepackage[T1]{fontenc}


\hypersetup{pdfstartview={FitH}}

\title{A Watch Using \texttt{Subleq}}
\author{Baptiste Lefebvre, Li-yao Xia, Antonin Delpeuch}
\date{\begin{tikzpicture}[scale=0.25, every node/.style={transform shape}]
        \node at (0,0) (n0) {};
        \SSGNb[0.75cm]{n0}{2}
        \node at (1,0) (n1) {};
        \SSGNb[0.75cm]{n1}{0}
        \node at (2,0) (n2) {};
        \SSGNb[0.75cm]{n2}{1}
        \node at (3,0) (n3) {};
        \SSGNb[0.75cm]{n3}{3}
        \node at (4.5,0) (n4) {};
        \SSGNb[0.75cm]{n4}{0}
        \node at (5.5,0) (n5) {};
        \SSGNb[0.75cm]{n5}{1}
        \node at (7,0) (n6) {};
        \SSGNb[0.75cm]{n6}{2}
        \node at (8,0) (n7) {};
        \SSGNb[0.75cm]{n7}{2}
        \node at (9.5,0) (n8) {};
        \SSGNb[0.75cm]{n8}{1}
        \node at (10.5,0) (n9) {};
        \SSGNb[0.75cm]{n9}{5}
        \node at (12,0) (n10) {};
        \SSGNb[0.75cm]{n10}{5}
        \node at (13,0) (n11) {};
        \SSGNb[0.75cm]{n11}{5}
  \end{tikzpicture}}


\begin{document}

\makeatletter

\begin{frame}
    \maketitle
\end{frame}

\AtBeginSection[]
{
  \begin{frame}<beamer>
    \tableofcontents[currentsection]
  \end{frame}
}

\begin{frame}{Project}
  Design a microprocessor for a digital watch.
  \begin{itemize}
    \item Minijazz compiler (provided)
    \item Logic circuit simulator
    \item Compiler-interpret for a custom assembly language
  \end{itemize}
\end{frame}

\section{The \texttt{Subleq} language}
\subsection{Motivation}
\begin{frame}
  \frametitle{Motivation}
  \begin{itemize}
    \item We already discovered the MIPS architecture previously
      in the compilation course.
    \item<2-> Design an original processor.
    \item<3-> The R in RISC may be misleading. The O in OISC is not.
    \item<4-> \texttt{Subleq} is simple and fun !
  \end{itemize}
\end{frame}

\subsection{An exotic architecture}
\begin{frame}{An exotic architecture}
  \begin{itemize}
    \item OISC. One Instruction Set Computer.
    \item<2-> \texttt{Subleq} achieves Turing-completeness
      with one instruction.
    \item<3-> A short documentation. %and that's all you will ever need
  \end{itemize}
\end{frame}

\begin{frame}[fragile]
    \frametitle{\texttt{Subleq}}
    The instruction set is made of only one instruction with three
    operands which represent memory addresses :
    \\[11pt]
    \begin{verbatim}
      A B C
    \end{verbatim}
    \pause
    which means :

    \begin{alltt}
       B \(\leftarrow\) *B - *A
      {\bf if} B \(\leq\) 0
      {\bf then} go to B
      {\bf else} go to next instruction
    \end{alltt}

\end{frame}

\begin{frame}
  \frametitle{\texttt{Subleq}}
  \begin{itemize}
    \item \texttt{Subleq} code is a sequence of integers% which initialize
      %a contiguous memory block starting from address 0.
    \item<2-> A program counter keeps track of the next instruction to execute.
    \item<3-> Data and instructions in the same place.
  \end{itemize}
\end{frame}

\subsection{Examples}
\begin{frame}[fragile]{Examples}
  \begin{alltt}
  3  4  5   8  8   8  3  0  -1     PC : 0\pause
  3  4  5   8  0   8  3  0  -1     PC : 5\pause
  3  4  5   9  0   8  3  0  -1     PC : 0\pause
  3  4  5   9 -9   8  3  0  -1     ...\pause
  3  4  5  10 -9   8  3  0  -1
  ...
  \end{alltt}
  \begin{itemize}
    \item<6-> An instruction does not need to be aligned.
    \item<7-> Fun fact : You can write subleq with only a numerical keypad.
  \end{itemize}
\end{frame}

\begin{frame}[fragile]
    \frametitle{Examples}
Syntactic sugar : Labels, semi-colon (address of the following cell).
Common instructions can be written with \texttt{subleq}, for instance :

\begin{columns}[t]
  \begin{column}[T]{3cm}
    \begin{alltt}
{\bf move} a b :
   a Z ;
   b b ;
   Z b ;
   Z Z ;

{\bf add} a b c :
   c Z ;
   b Z ;
   a a ;
   Z a ;
   Z Z ;
    \end{alltt}
  \end{column}
  \begin{column}[T]{3cm}
    \begin{alltt}
{\bf beqz} a lbl :
   	Z a L1
   	Z Z L3
 L1:
    a Z L2
    Z Z L3
 L2:
    Z Z lbl
 L3:

{\bf j} lbl :
    Z Z lbl
    \end{alltt}
  \end{column}
  \end{columns}
\end{frame}


\section{Chip architecture}
\subsection{Technical specifications}

\begin{frame}{Technical specifications}
  Digital watch.
  \begin{itemize}
    \item Memory : $2^{15}$ 2-byte words.
    \item Speed : 1024 Hz (200 IPS)
  \end{itemize}
\end{frame}

\subsection{Instruction cycle}

\begin{frame}[fragile]
  \frametitle{Instruction cycle}
  Steps in decoding the instruction \verb=A B C= :
      \begin{enumerate}
        \item<2-> \verb=Read A=
        \item<3-> \verb=Read B=
        \item<4-> \verb=Read *A=
        \item<5-> \verb=Read *B=
        \item<6->
          \begin{alltt}
Read C
If VB - VA \(\leq\) 0
  then set PC to C
else set PC to PC+3
          \end{alltt}
      \end{enumerate}
\end{frame}

\subsection{Chip layout}
\begin{frame}
    \frametitle{Layout}
    \centering
    
\tikzstyle{operateur} =
[style={circle,fill=gray!20,draw,font=\sffamily\large\bfseries}]
\tikzstyle{not} = [draw,style={circle,fill=gray!20},minimum size=0.0cm]
\tikzstyle{reg} = [style={fill=gray!20,draw,thick,minimum size=0.8cm}]
\tikzstyle{boite} = [minimum height=2cm, minimum width=2cm,rectangle,
					style={fill=gray!20,draw,thick}]
\tikzstyle{ptiteboite} = [minimum height=1.5cm, minimum width=1.5cm,rectangle,
					style={fill=gray!20,draw,thick}]
\tikzstyle{rambox} = [minimum height=7cm, minimum width=2cm,rectangle,
					style={fill=gray!20,draw,thick}]
\tikzstyle{muxv} = [minimum height=1.6cm, minimum width=0.5cm,rectangle,
	               style={fill=gray!20,draw,thick}]
\tikzstyle{mux} = [minimum height=0.5cm, minimum width=1.6cm,rectangle,
	               style={fill=gray!20,draw,thick}]
\tikzstyle{muxb} = [node distance=0.4cm,style={minimum size=0.5cm}]
\tikzstyle{vide} = []
\tikzstyle{noeud} = [style={circle,fill=black,minimum size=0cm}]
\tikzstyle{fil} = [->,style={thick}]
\tikzstyle{filb} = [style={thick}]
\tikzstyle{ecrit} = [font=\sffamily\small\bfseries]
\tikzstyle{exterieur} = [dashed, draw]
\tikzstyle{hypot} = [dashed]

\begin{tikzpicture}[auto, node distance=3.5cm,>=latex', scale = 0.75, every node/.style={transform shape}]

\node[boite] (muxers) {Muxers};

\node[boite, right of=muxers] (memctrl) {Memory controller};
\node[boite, above right of=memctrl] (ram) {RAM};
\node[ptiteboite, below right of=memctrl] (dd) {DD};
\node[ptiteboite, right of=dd, node distance=2.5cm] (bcd) {BCD};

\node[reg, above of=muxers] (pc) {PC};
\node[reg, below of=muxers] (b) {B};
\node[reg, left of=b,node distance=1cm] (a) {A};
\node[reg, right of=b,node distance=1cm] (vb) {VB};
\node[boite, left of=muxers] (sched) {Scheduler};
\node[operateur, below of=b, node distance=2cm] (sub) {sub};
\node[operateur, left of=sub] (leqz) {leqz};

\draw[fil] (sched) -- (muxers);
\draw[filb] (muxers) -- (memctrl);
\draw[filb] (pc) -- (muxers);
\draw[filb] (a) |- ($(a) + (0.8cm,0.6cm)$) -| ($(muxers)+(-0.2cm,-1cm)$);
\draw[filb] (b) -- (muxers);
\draw[filb] (vb) |- ($(vb) + (-0.8cm,0.6cm)$) -| ($(muxers)+(0.2cm,-1cm)$);

\draw[fil] (memctrl) |- (dd);
\draw[fil] (dd) -- (bcd);
\draw[filb] (memctrl) |- (ram);

\draw[fil] (a) |- ($(a)+(0.4cm,-0.6cm)$) -| ($(sub)+(-0.4cm,0.4cm)$);
\draw[fil] (vb) |- ($(vb)+(-0.4cm,-0.6cm)$) -| ($(sub)+(0.4cm,0.4cm)$);
\draw[fil] (sub) -- (leqz);
\draw[fil] (leqz) -| ($(leqz)+(1.8cm,3cm)$) -| ($(muxers)+(-0.4cm,-1cm)$);
\draw[fil] (leqz) |- ($(leqz)+(1.8cm,3.2cm)$) -| ($(muxers)+(-0.6cm,-1cm)$);


\end{tikzpicture}


\end{frame}

\begin{frame}{Possible improvement}
  \begin{itemize}
    \item A parallel addressing RAM : One clock tick = one instruction.
  \end{itemize}
\end{frame}


\section{Implementation details}

\subsection{Memory management}

\begin{frame}{Memory management}
  \begin{itemize}
    \item Pure \texttt{Subleq} has no output feature and memory is
      unbounded in theory.
    \item<2-> Our solution : immediate access to some part of the memory
      in the circuit.
    \item<3-> Negative addresses will be "virtual memory cells".
  \end{itemize}
\end{frame}

\begin{frame}{Memory management}
	\centering
	

\tikzstyle{cell} = [minimum height=0.6cm, minimum width=3cm,rectangle,style={fill=gray!20,draw,thick}]
\tikzstyle{muxn} = [minimum height=3.5cm, minimum width=0.6cm,rectangle,style={fill=gray!20,draw,thick}]
\tikzstyle{fil} = [->,style={thick}]
\tikzstyle{filb} = [style={thick}]

\begin{tikzpicture}[auto, scale = 0.6]
% Virtual cells	
	\node[cell] at (0cm,5cm) (y) {y};
	\node[cell] at (0cm,4cm) (m) {m};
	\node[cell] at (0cm,3cm) (d) {d};
	\node[cell] at (0cm,2cm) (h) {h};
	\node[cell] at (0cm,1cm) (min) {min};

% RAM cells
	\node[cell] at (0cm,0cm) (sec) {sec};
	\node[cell] at (0cm,-2cm) (Ram0) {0};
	\node[cell] at (0cm,-3cm) (Ram1) {1};
	\node[cell] at (0cm,-4cm) (dots) {...};
	\node[cell] at (0cm,-5cm) (Ram2n) {\(2^{address-	size}\)};

% Mux for virtual cells
	\node[muxn] at (-5cm,2.5cm) (muxn) {};
	\node at (-4.7cm,5cm) (muxny) {};
	\node at (-4.7cm,4cm) (muxnm) {};
	\node at (-4.7cm,3cm) (muxnd) {};
	\node at (-4.7cm,2cm) (muxnh) {};
	\node at (-4.7cm,1cm) (muxnmin) {};
	\node at (-4.7cm,0cm) (muxnsec) {};
	\draw[fil] (muxny) -- (y);
	\draw[fil] (muxnm) -- (m);
	\draw[fil] (muxnd) -- (d);
	\draw[fil] (muxnh) -- (h);
	\draw[fil] (muxnmin) -- (min);
	\draw[fil] (muxnsec) -- (sec);

% Demux for virtual cells
	\node[muxn] at (5cm,2.5cm) (demuxn) {};
	\node at (4.7cm,5cm) (demuxny) {};
	\node at (4.7cm,4cm) (demuxnm) {};
	\node at (4.7cm,3cm) (demuxnd) {};
	\node at (4.7cm,2cm) (demuxnh) {};
	\node at (4.7cm,1cm) (demuxnmin) {};
	\node at (4.7cm,0cm) (demuxnsec) {};
	\draw[fil] (y) -- (demuxny);
	\draw[fil] (m) -- (demuxnm);
	\draw[fil] (d) -- (demuxnd);
	\draw[fil] (h) -- (demuxnh);
	\draw[fil] (min) -- (demuxnmin);
	\draw[fil] (sec) -- (demuxnsec);

\end{tikzpicture}

\end{frame}

\subsection{7-segments display}

\begin{frame}
  \frametitle{7-segments display}
  The conversion from binary values to 7-segments display is built
  in the chip, in two steps :
  \begin{enumerate}
    \item<2-> Binary $\rightarrow$ Binary Coded Decimal (BCD)

          \texttt{101010} $\rightarrow$ \texttt{0100}.\texttt{0010}
    \item<3-> BCD $\rightarrow$ 7-segments

        \texttt{0100}.\texttt{0010} $\rightarrow$
        \texttt{1100110}.\texttt{0111011}
\end{enumerate}
\onslide<4->
The first step is done with the \emph{double dabble} algorithm,
the second step using a circuit derived from the desired truth table.

\end{frame}

\begin{frame}
    \frametitle{Double dabble}

    How to multiply by 2 a number written in BCD ?
    \pause

    If the digits are all less than 5, one shift is enough.
    Otherwise we propagate a carry by adding 3 to the digit before shifting it.
    \pause

    \begin{columns}[t]
        \begin{column}[T]{5cm}
            \vspace{1cm}
    \begin{tikzpicture}
        \node at (-0.75,0.5) {2};
        \node at (0.75,0.5) {3};
        \node (init) {\texttt{0 0 1 0 0 0 1 1}};
        \node at (0,-0.4) (next) {\texttt{0 1 0 0 0 1 1 0}};
        \draw (0,0.2) -- (0,-0.6);
        \node at (-0.75,-0.9) {4};
        \node at (0.75,-0.9) {6};
        \node at (2.1,-0.4) (sh) {shift};
    \end{tikzpicture}
        \end{column}
        \begin{column}[T]{5cm}
            \vspace{1cm}
  \begin{tikzpicture}
        \node at (-0.75,0.5) {4};
        \node at (0.75,0.5) {6};
        \node (init) {\texttt{0 1 0 0 0 1 1 0}};
        \node at (0,-0.4) (next) {\texttt{0 1 0 0 1 0 0 1}};
        \node at (0,-0.8) (next) {\texttt{1 0 0 1 0 0 1 0}};
        \draw (0,0.2) -- (0,-1);
        \node at (-0.75,-1.3) {9};
        \node at (0.75,-1.3) {2};
        \node at (2.2,-0.4) (add3) {add 3};
        \node at (2.1,-0.8) (sh) {shift};
    \end{tikzpicture}

        \end{column}
    \end{columns}

    We add 3 because $2*(5 + 3) = 16$.
\end{frame}

\begin{frame}
  \frametitle{Implementation}

  

\tikzstyle{add} = [minimum height=0.4cm, minimum width=1.5cm,rectangle,
					style={fill=gray!20,draw,thick}]
\tikzstyle{fil} = [style={thick}]

\centering
\begin{tikzpicture}[scale=1.2, node/.style={transform shape}] 

\node at (-1.575,0) (i0) {\texttt{0}};
\node at (-1.925,0) (i0) {\texttt{0}};
\foreach \x in {7, ..., 0} {
\node at ($(-1.575+8*0.35,0)+\x*(-0.35,0)$) (ip\x) {$i_\x$};
}

\node[add] at (-1,-0.75) (a1) {};
\node[add] at (-0.65,-1.5) (a2) {};
\node[add] at (-0.3,-2.25) (a3) {};
\node[add] at (0.05,-3) (a4) {};
\node[add] at (0.4,-3.75) (a5) {};
\node[add] at (-1.35,-3) (a6) {};
\node[add] at (-1,-3.75) (a7) {};

\draw[fil] (-1.525,-0.25) -- ($(a1)+(-0.525,0.2)$);
\draw[fil] (-1.175,-0.25) -- ($(a1)+(-0.175,0.2)$);
\draw[fil] (-0.825,-0.25) -- ($(a1)+(0.175,0.2)$);
\draw[fil] (-0.475,-0.25) -- ($(a1)+(0.525,0.2)$);
\draw[fil] (-0.15,-0.25) -- ($(a2)+(0.525,0.2)$);
\draw[fil] (0.2,-0.25) -- ($(a3)+(0.525,0.2)$);
\draw[fil] (0.55,-0.25) -- ($(a4)+(0.525,0.2)$);
\draw[fil] (0.9,-0.25) -- ($(a5)+(0.525,0.2)$);
\draw[fil] (1.25,-0.25) -- ++(0,-4.2);

\draw[fil] ($(a1)+(0.525,-0.2)$) -- ($(a2)+(0.175,0.2)$);
\draw[fil] ($(a2)+(0.525,-0.2)$) -- ($(a3)+(0.175,0.2)$);
\draw[fil] ($(a3)+(0.525,-0.2)$) -- ($(a4)+(0.175,0.2)$);
\draw[fil] ($(a4)+(0.525,-0.2)$) -- ($(a5)+(0.175,0.2)$);
\draw[fil] ($(a5)+(0.525,-0.2)$) -- ++(0,-0.5);

\draw[fil] ($(a1)+(0.175,-0.2)$) -- ($(a2)+(-0.175,0.2)$);
\draw[fil] ($(a2)+(0.175,-0.2)$) -- ($(a3)+(-0.175,0.2)$);
\draw[fil] ($(a3)+(0.175,-0.2)$) -- ($(a4)+(-0.175,0.2)$);
\draw[fil] ($(a4)+(0.175,-0.2)$) -- ($(a5)+(-0.175,0.2)$);
\draw[fil] ($(a5)+(0.175,-0.2)$) -- ++(0,-0.5);

\draw[fil] ($(a1)+(-0.175,-0.2)$) -- ($(a2)+(-0.525,0.2)$);
\draw[fil] ($(a2)+(-0.175,-0.2)$) -- ($(a3)+(-0.525,0.2)$);
\draw[fil] ($(a3)+(-0.175,-0.2)$) -- ($(a4)+(-0.525,0.2)$);
\draw[fil] ($(a4)+(-0.175,-0.2)$) -- ($(a5)+(-0.525,0.2)$);
\draw[fil] ($(a5)+(-0.175,-0.2)$) -- ++(0,-0.5);

\draw[fil] ($(a1)+(-0.525,-0.2)$) -- ($(a6)+(-0.175,0.2)$);
\draw[fil] ($(a2)+(-0.525,-0.2)$) -- ($(a6)+(0.175,0.2)$);
\draw[fil] ($(a3)+(-0.525,-0.2)$) -- ($(a6)+(0.525,0.2)$);
\draw[fil] (-1.925,-0.25) -- ($(a6)+(-0.525,0.2)$);

\draw[fil] ($(a6)+(-0.175,-0.2)$) -- ($(a7)+(-0.525,0.2)$);
\draw[fil] ($(a6)+(0.175,-0.2)$) -- ($(a7)+(-0.175,0.2)$);
\draw[fil] ($(a6)+(0.525,-0.2)$) -- ($(a7)+(0.175,0.2)$);
\draw[fil] ($(a4)+(-0.525,-0.2)$) -- ($(a7)+(0.525,0.2)$);

\draw[fil] ($(a7)+(-0.175,-0.2)$) -- ++(0,-0.5);
\draw[fil] ($(a7)+(0.175,-0.2)$) -- ++(0,-0.5);
\draw[fil] ($(a7)+(-0.525,-0.2)$) -- ++(0,-0.5);
\draw[fil] ($(a7)+(0.525,-0.2)$) -- ++(0,-0.5);

\draw[fil] ($(a5)+(-0.525,-0.2)$) -- ++(0,-0.5);
\draw[fil] ($(a6)+(-0.525,-0.2)$) -- ++(0,-1.25);

\foreach \x in {9, ..., 0} {
\node at ($(-1.910+9*0.35,-4.65)+\x*(-0.35,0)$) (op\x) {$o_\x$};
}
\end{tikzpicture}



\end{frame}

\subsection{Time management and programming}
\begin{frame}
  \frametitle{Time tracking and programming}
  \begin{itemize}
    \item 5 clock ticks per instruction.
    \item 1024 Hz (about 200 IPS)
    \item<2-> Up to the watch program to keep track of elapsed cycles.
    \item<3-> Unused cycles must be spent in a loop.
  \end{itemize}
\end{frame}

\begin{frame}[fragile]
  \frametitle{Time tracking and programming}
    \begin{verbatim}
...
start:
p10 cyclesleft main
0 0 start

main:
...
p40 cyclesleft ;
# reset minutes
-2 -2 ;
... # a sequence of 6 more instructions
    \end{verbatim}
\end{frame}

\begin{frame}
  \frametitle{Summary}
  \begin{itemize}
    \item One instruction. Learn a language in 5 minutes.
    \item<2-> Few logic gates. %because of the architecture choice
    \item<3-> Given a perfect clock, our watch should drift away from UTC
      by less than a second a year. (leap seconds)
    \item<4-> Valid display until December 31st, 4095
  \end{itemize}
\end{frame}


\end{document}


\documentclass[a4paper]{article}

\usepackage[french]{babel}
\usepackage[T1]{fontenc}
\usepackage[utf8]{inputenc}
\usepackage{fullpage}
\usepackage{pgf}
\usepackage{tikz}
\usetikzlibrary{arrows,automata}
\usetikzlibrary{positioning}

\begin{document}

\title{Architecture du processeur}
\author{Baptiste Lefebvre, Li-yao Xia, Antonin Delpeuch}
\date{\today}

\maketitle

\section{Architecture}

\subsection{Fonctionnement}

L'heure est représentée dans un compteur qui est incrémenté
périodiquement indépendemment de toute instruction.
On dispose également d'une file de priorité qui stocke des paires $(date,
adresse)$ : quand l'heure actuelle correspond à celle en tête de file, le
code à l'adresse correspondante est exécuté. Ce code est écrit dans le
langage assembleur décrit ci-dessous. Il est stocké dans une ROM.

Cette file est gérée directement au niveau matériel : le tri des éléments
est géré en inversant deux voisins quand c'est nécessaire. Ce tri se
parallélise et est donc plus rapide que les tris classiques.

Cette file permet de gérer les exceptions du calendrier : par exemple,
pour rajouter un jour les années bissextiles, on insère dans la file un
évènement à la date du 28 février de la prochaine année bissextile, qui permettera de
mettre à jour la date correctement et d'insérer un nouvel évènement pour
le prochain 28 février d'une année bissextile.

\subsection{Mémoire}

On a une RAM de petite taille (par exemple pour stocker les résultats
intermédiaires des calculs).

Les registres sont :
\begin{itemize}
\item Les registres d'entrée (accessibles en lecture uniquement)
\item Les registres de sortie (pour l'afficheur 7 segments)
\item Les registres de calcul
\item Un registre pour l'adresse de l'instruction courante
\end{itemize}

\section{Jeu d'instructions}

On a deux types de données : des entiers et des dates, les durées étant
stockées dans l'un des deux.

\begin{itemize}
\item Lire dans la RAM
\item Écrire dans la RAM
\item Saut (in)conditionnel
\item (Pas de \texttt{jal})
\item Opérations arithmétiques :
\begin{itemize}
\item Addition (entier $+$ entier, date $+$ entier, date $+$ date)
\item Soustraction (idem)
\item Multiplication (entier $\times$  entier, entier $\times$ date)
\item Division
\item Modulo
\item Comparaison
\end{itemize}
\item Conversion entier / date
\item exit
\item Affichage arbitraire sur l'écran de la montre ($n$ bits)
\item Insertion d'un élément dans la file
\item Un \texttt{nop} !
\end{itemize}

\end{document}


